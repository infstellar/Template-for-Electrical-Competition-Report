\documentclass[UTF-8,zihao=-4]{ctexart}

% --- 基础宏包 ---
\usepackage{geometry} % 设置页边距
\usepackage{graphicx} % 插入图片
\usepackage{amsmath} % 数学公式
\usepackage{amssymb} % 数学符号
\usepackage{float} % 控制图表浮动位置, [H]表示强制在此处
\usepackage{hyperref} % 生成超链接
\usepackage{listings} % 插入代码
\usepackage{xcolor} % 定义颜色
\usepackage{booktabs} % 三线表宏包
\usepackage{fancyhdr} % 自定义页眉页脚 (可选)

\setCJKmainfont{SimSun}
% --- 页面设置 ---
\geometry{a4paper, left=2.5cm, right=2.5cm, top=3cm, bottom=2.5cm}
\setlength{\baselineskip}{22pt} % 设置固定行距为 22pt

% --- 超链接设置 ---
\hypersetup{
    colorlinks=true,
    linkcolor=black,
    filecolor=magenta,      
    urlcolor=blue,
    citecolor=black,
}

% --- 页眉页脚设置 ---
\pagestyle{fancy}                 % 启用 fancy 样式
\fancyhf{}                        % 清空所有页眉页脚
\renewcommand{\headrulewidth}{0pt} % 去掉页眉的横线
\rfoot{\thepage}                  % 在页脚的右侧放置页码

% --- 代码块样式定义 ---
\definecolor{codegreen}{rgb}{0,0.6,0}
\definecolor{codegray}{rgb}{0.5,0.5,0.5}
\definecolor{codepurple}{rgb}{0.58,0,0.82}
\definecolor{backcolour}{rgb}{0.95,0.95,0.92}

\lstdefinestyle{mystyle}{
    backgroundcolor=\color{backcolour},   
    commentstyle=\color{codegreen},
    keywordstyle=\color{magenta},
    numberstyle=\tiny\color{codegray},
    stringstyle=\color{codepurple},
    basicstyle=\ttfamily\footnotesize,
    breakatwhitespace=false,         
    breaklines=true,                 
    captionpos=b,                    
    keepspaces=true,                 
    numbers=left,                    
    numbersep=5pt,                  
    showspaces=false,                
    showstringspaces=false,
    showtabs=false,                  
    tabsize=2
}
\lstset{style=mystyle}

%================================================
%==============  文档开始  ======================
%================================================
\begin{document}

% --- 标题页 ---
\begin{titlepage}
    \centering
    \vspace*{2cm}
    
    {\Huge \bfseries [年份]年全国大学生电子设计竞赛} % <-- 修改年份
    
    \vspace{2.5cm}
    
    {\huge \bfseries [题目编号] 题:[题目名称]} % <-- 修改题号和题目
    
    \vspace{1.5cm}
    
    {\Large \bfseries 设 计 报 告}
    
    \vspace{2cm}

    \begin{figure}[H]
        \centering
        \includegraphics[width=0.5\textwidth]{../pic/电赛logo.png}
    \end{figure}

    \vspace{6.5cm}

    
    {\large \today}
    
\end{titlepage}

% --- 摘要 ---
% \begin{abstract}
% 这里由于LaTex的摘要格式会导致“摘要”二字较小,这里选用quote环境来解决
\section*{摘要}
\begin{quote}
    \noindent
    % [ 摘要内容模板 ]
    % 第一部分:概述系统。格式:本作品设计并制作了一个基于[核心控制器型号,如TI MSPM0G3507]的[系统名称]。系统采用[总体技术方案,如:XXX结构],集成了[主要传感器/模块,如:XXX传感器、XXX模块]等,实现了赛题要求的[核心功能,如:自动XXX与XXX]功能。
    % 第二部分:阐述算法。格式:在算法方面,我们采用了[核心算法1,如:PID控制]与[核心算法2,如:XXX滤波]等关键技术。我们使用[具体算法描述]对[处理对象]进行处理,得到[中间结果],并结合[控制器名称]实现了[闭环控制目标],完成了题目要求的各项任务。
    % 第三部分:总结陈述。格式:经测试,本作品能够稳定、可靠地完成指定任务,各项指标均满足或优于题目要求。
    [ 此处填写摘要内容... ]

    \vspace{1cm}
    \noindent
    \textbf{关键词:} [关键词一];[关键词二];[关键词三];[关键词四] % <-- 修改关键词
% \end{abstract}
\end{quote}

\newpage
\tableofcontents
\newpage

%================================================
%==============  正文部分  ======================
%================================================

\section{系统方案设计}
    % 本章节对应评分标准中的“系统方案”,需要清晰描述整体设计思路和选择。
\subsection{系统方案描述}
    % 简要介绍系统的总体构成,说明各个模块(主控、电源、传感器、驱动、执行机构)的功能和它们之间的关系。
    本系统以 MSPM0G3507 微控制器为核心,主要由12v转3.3v/5v分压电路、树莓派Zero 2W模块、OpenMV H7摄像头模块、TB6612电机驱动模块,513X马达,DS3115MG舵机驱动的二维云台等部分组成。系统整体功能框图如图\ref{fig:system_block}所示。
    
\begin{figure}[H]
	\centering
	% \includegraphics[width=0.8\linewidth]{../pic/your_block_diagram.png} % <-- 替换为你的系统框图文件
	\caption{系统整体功能框图}
	\label{fig:system_block}
\end{figure}

\subsection{方案论证与选择}
    % 本部分非常重要,需要对关键器件和方案进行比较和选择,并说明理由。
\subsubsection{主控制器件的论证与选择}
    % [ 论证模板 ]
    % 方案一:[方案一名称],如[具体型号]。优点:[优点]。缺点:[缺点]。
    % 方案二:[方案二名称],如[具体型号]。优点:[优点]。缺点:[缺点]。
    % 结论:综合考虑本题对[性能指标,如计算能力]和[资源需求,如外设]的需求,选择[最终方案]作为主控芯片。
    方案一:基于STM32F103C8T6的方案。STM32芯片成熟稳定,社区支持丰富,外设资源丰富,但开发迭代效率不高,处理能力相对较弱,无法满足高性能需求。
    方案二:基于树莓派Zero 2W的方案。优点:处理能力强大,支持Python等高级语言,易于开发。缺点:功耗较高,实时性较差。
    结论:综合考虑本题对处理能力和资源需求的要求,选择树莓派Zero 2W作为主控芯片。

\subsubsection{循迹模块的论证与选择}


\subsubsection{机器视觉模块的论证与选择}
    通用的机器视觉方案有基于树莓派的OpenCV方案和基于OpenMV的方案。OpenMV易于开发,独立运算,但是上限较低;树莓派+OpenCV方案功能强大,但在树莓派上运行OpenCV的资源开销较大。综合考虑本题对机器视觉的要求,选择OpenMV H7作为机器视觉模块可以基本满足检测目标靶纸的需求。OpenMV与树莓派使用USART通信,树莓派通过串口获取OpenMV处理后的数据,减小运算量。
    % 针对系统中的关键技术点或核心模块进行选型比较。可添加多个子小节。
    % [ 论证模板 ]
    % 1. [关键模块一名称]
    % 方案一:[方案一描述]。优点:[优点]。缺点:[缺点]。
    % 方案二:[方案二描述]。优点:[优点]。缺点:[缺点]。
    % 结论:结合[赛题要求]和[实际情况],我们选择[最终方案]。
\subsubsection{云台模块控制方法的论证}
    方案一:仅使用摄像头识别的靶纸位置进行云台控制。优点:实现简单,成本低。缺点:OpenMV的更新频率在33Hz左右,在剧烈速度变化(急加速,转向)时,可能无法及时调整云台角度,导致跟踪不准确。
    方案二:使用树莓派获取IMU传感器数据,结合摄像头识别的靶纸位置进行云台控制。优点:IMU传感器可以提供实时的姿态信息,能够更快地调整云台角度,提高跟踪精度。缺点:增加了系统复杂度和成本。

\section{系统理论分析与计算}
    % 本章节对应评分标准中的“理论分析”,需要展示核心算法的原理和必要的计算。
\subsection{核心理论/模型分析}
    % 描述系统所基于的核心理论模型,如控制理论、通信原理等。
    % 给出必要的原理图和数学模型。
    [ 此处填写核心理论分析... ]
    
    % \begin{figure}[H]
    %     \centering
    %     % \includegraphics[width=0.8\textwidth]{pic/your_model.png} % <-- 替换为你的理论模型图
    %     \caption{系统理论模型图}
    %     \label{fig:theory_model}
    % \end{figure}
    
\subsection{关键算法分析}
    % 详细介绍您所使用的关键算法。
    % 例如:PID控制器设计、滤波器设计、数据融合算法、通信协议等。
    % 给出关键的数学公式,并解释公式中各符号的含义。
    [ 此处填写关键算法分析... ]
    例如,某算法的关键公式如式\ref{eq:example}所示:
    \begin{equation} \label{eq:example}
        y(t) = K_p e(t) + K_i \int_0^t e(\tau)d\tau + K_d \frac{de(t)}{dt}
    \end{equation}
    其中,$e(t)$ 表示误差... % <-- 对公式和符号进行解释

\subsection{相关参数计算}
    % 根据题目要求和系统设计,进行必要的参数计算。
    % 例如:根据场地尺寸计算运动参数;根据器件手册计算电路参数等。
    % 这些计算结果是系统设计和实现的重要依据。
    [ 此处填写相关参数的计算过程和结果... ]
    
\section{电路与程序设计}
    % 本章节对应评分标准中的“电路与程序设计”。
\subsection{电路设计}
    % 分模块展示核心电路图,并简要说明设计思路。可添加多个子小节。
\subsubsection{核心模块电路(一)}
    % [ 此处填写第一个核心模块的电路设计说明,例如电源模块 ]
    % \begin{figure}[H]
    %     \centering
    %     % \includegraphics[width=0.7\textwidth]{pic/your_circuit_1.png} % <-- 替换为你的电路原理图
    %     \caption{模块一电路原理图}
    %     \label{fig:circuit_1}
    % \end{figure}
    
\subsubsection{核心模块电路(二)}
    % [ 此处填写第二个核心模块的电路设计说明,例如驱动模块 ]
    % \begin{figure}[H]
    %     \centering
    %     % \includegraphics[width=0.7\textwidth]{pic/your_circuit_2.png} % <-- 替换为你的电路原理图
    %     \caption{模块二电路原理图}
    %     \label{fig:circuit_2}
    % \end{figure}

\subsection{程序设计}
\subsubsection{主程序设计思路}
    % 描述主程序结构,建议使用状态机思想。
    % 例如:系统上电后进行初始化,包括GPIO、定时器、串口、I2C等。然后进入待机状态,等待启动信号。
    % 启动后,根据当前任务进入不同的主状态。在每个主状态下,通过子状态机来完成具体动作。
    % 定时中断用于周期性地处理关键任务,如读取传感器、更新控制器状态等。
    [ 此处填写主程序设计思路... ]

\subsubsection{程序流程图}
    % \begin{figure}[H]
    %     \centering
    %     % \includegraphics[width=0.6\textwidth]{pic/your_flowchart.png} % <-- 替换为你的主程序流程图
    %     \caption{主程序流程图}
    %     \label{fig:main_flowchart}
    % \end{figure}
    
\subsubsection{核心代码片段 (可选)}
% \begin{lstlisting}[language={[编程语言]}, caption={[代码功能描述]}, label={lst:code_example}]
% // 在这里粘贴您的核心功能的代码
% // 例如:
% float function_example(float input) {
%     // ...
%     return output;
% }
% \end{lstlisting}

\section{测试方案与结果分析}
    % 本章节对应评分标准中的“测试方案与测试结果”。
\subsection{测试方案}
    \begin{itemize}
        \item \textbf{测试环境:} [描述测试场地、环境条件等,需与题目要求一致]。
        \item \textbf{测试仪器:} [列出所用仪器,如示波器、万用表、频谱仪、秒表、卷尺等]。
        \item \textbf{测试方法:} 针对赛题要求的各项测试指标,分别进行测试。描述清楚每个指标的具体测试步骤、操作方法和数据记录方式。
    \end{itemize}

\subsection{测试结果与数据}
    % 使用表格清晰地展示测试结果,并与题目要求对比。
    \begin{table}[H]
        \centering
        \caption{测试项(1) [测试项名称] (要求: [指标要求])}
        \label{tab:task1}
        \begin{tabular}{cccc}
            \toprule
            测试序号 & 测试结果 & 是否满足要求 & 备注 \\
            \midrule
            1 & & & \\
            2 & & & \\
            3 & & & \\
            \bottomrule
        \end{tabular}
    \end{table}

    \begin{table}[H]
        \centering
        \caption{测试项(2) [测试项名称] (要求: [指标要求])}
        \label{tab:task2}
        \begin{tabular}{cccc}
            \toprule
            测试序号 & 测试结果 & 是否满足要求 & 备注 \\
            \midrule
            1 & & & \\
            2 & & & \\
            3 & & & \\
            \bottomrule
        \end{tabular}
    \end{table}
    
    % ... 根据需要为更多测试项创建表格 ...

\subsection{误差/性能分析}
    % 分析实际运行中可能出现的误差来源或性能瓶颈,并提出改进方向。
    \begin{enumerate}
        \item \textbf{[误差/瓶颈来源一]:} [详细描述该问题及其对系统的影响]。改进方法:[提出具体的改进措施或优化方向]。
        \item \textbf{[误差/瓶颈来源二]:} [详细描述该问题及其对系统的影响]。改进方法:[提出具体的改进措施或优化方向]。
        \item \textbf{[误差/瓶颈来源三]:} [详细描述该问题及其对系统的影响]。改进方法:[提出具体的改进措施或优化方向]。
    \end{enumerate}

% --- 参考文献 ---
\begin{thebibliography}{99}
    \bibitem{ref1} [作者]. [文献标题][J/M]. [期刊/出版社], [年份].
    \bibitem{ref2} [作者]. [书籍名称][M]. [出版社], [年份].
    % 在这里列出您参考过的文献、书籍、技术手册等
\end{thebibliography}

% --- 附录 ---
\appendix
\section{主要元器件清单}
\begin{table}[H]
    \centering
    \caption{硬件BOM表}
    \label{tab:bom}
    \begin{tabular}{lll}
        \toprule
        元器件名称 & 型号 & 数量 \\
        \midrule
        微控制器 & [例如: TI MSPM0G3507] & 1 \\
        传感器模块 & [例如: MPU6050] & 1 \\
        驱动芯片 & [例如: DRV8701] & 1 \\
        ... & ... & ... \\ % <-- 替换为您的元器件
        \bottomrule
    \end{tabular}
\end{table}



\end{document}