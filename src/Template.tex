\documentclass[UTF-8,zihao=-4]{ctexart}

% --- 基础宏包 ---
\usepackage{geometry} % 设置页边距
\usepackage{graphicx} % 插入图片
\usepackage{amsmath} % 数学公式
\usepackage{amssymb} % 数学符号
\usepackage{float} % 控制图表浮动位置, [H]表示强制在此处
\usepackage{hyperref} % 生成超链接
\usepackage{listings} % 插入代码
\usepackage{xcolor} % 定义颜色
\usepackage{booktabs} % 三线表宏包
\usepackage{fancyhdr} % 自定义页眉页脚 (可选)

\setCJKmainfont{SimSun}
% --- 页面设置 ---
\geometry{a4paper, left=2.5cm, right=2.5cm, top=3cm, bottom=2.5cm}
\setlength{\baselineskip}{22pt} % 设置固定行距为 22pt

% --- 超链接设置 ---
\hypersetup{
    colorlinks=true,
    linkcolor=black,
    filecolor=magenta,      
    urlcolor=blue,
    citecolor=black,
}

% --- 页眉页脚设置 ---
\pagestyle{fancy}                 % 启用 fancy 样式
\fancyhf{}                        % 清空所有页眉页脚
\renewcommand{\headrulewidth}{0pt} % 去掉页眉的横线
\rfoot{\thepage}                  % 在页脚的右侧放置页码

% --- 代码块样式定义 ---
\definecolor{codegreen}{rgb}{0,0.6,0}
\definecolor{codegray}{rgb}{0.5,0.5,0.5}
\definecolor{codepurple}{rgb}{0.58,0,0.82}
\definecolor{backcolour}{rgb}{0.95,0.95,0.92}

\lstdefinestyle{mystyle}{
    backgroundcolor=\color{backcolour},   
    commentstyle=\color{codegreen},
    keywordstyle=\color{magenta},
    numberstyle=\tiny\color{codegray},
    stringstyle=\color{codepurple},
    basicstyle=\ttfamily\footnotesize,
    breakatwhitespace=false,         
    breaklines=true,                 
    captionpos=b,                    
    keepspaces=true,                 
    numbers=left,                    
    numbersep=5pt,                  
    showspaces=false,                
    showstringspaces=false,
    showtabs=false,                  
    tabsize=2
}
\lstset{style=mystyle}

%================================================
%==============  文档开始  ======================
%================================================
\begin{document}

% --- 标题页 ---
\begin{titlepage}
    \centering
    \vspace*{2cm}
    
    {\Huge \bfseries [2025]年全国大学生电子设计竞赛} % <-- 修改年份
    
    \vspace{2.5cm}
    
    {\huge \bfseries [E] 题:[简易自行瞄准装置]} % <-- 修改题号和题目
    
    \vspace{1.5cm}
    
    {\Large \bfseries 设 计 报 告}
    
    \vspace{2cm}

    \begin{figure}[H]
        \centering
        \includegraphics[width=0.5\textwidth]{../pic/电赛logo.png}
    \end{figure}

    \vspace{6.5cm}

    
    {\large \today}
    
\end{titlepage}

% --- 摘要 ---
% \begin{abstract}
% 这里由于LaTex的摘要格式会导致“摘要”二字较小,这里选用quote环境来解决
\section*{摘要}
\begin{quote}
    \noindent
    % [ 摘要内容模板 ]
    % 第一部分:概述系统。格式:本作品设计并制作了一个基于[核心控制器型号,如TI MSPM0G3507]的[系统名称]。系统采用[总体技术方案,如:XXX结构],集成了[主要传感器/模块,如:XXX传感器、XXX模块]等,实现了赛题要求的[核心功能,如:自动XXX与XXX]功能。
    % 第二部分:阐述算法。格式:在算法方面,我们采用了[核心算法1,如:PID控制]与[核心算法2,如:XXX滤波]等关键技术。我们使用[具体算法描述]对[处理对象]进行处理,得到[中间结果],并结合[控制器名称]实现了[闭环控制目标],完成了题目要求的各项任务。
    % 第三部分:总结陈述。格式:经测试,本作品能够稳定、可靠地完成指定任务,各项指标均满足或优于题目要求。
    本系统以 MSPM0G3507 微控制器为核心,主要由12v转3.3v/5v分压电路、树莓派Zero 2W模块、OpenMV H7摄像头模块、TB6612电机驱动模块,513X马达,DS3115MG舵机驱动的二维云台等部分组成。
    \\自动寻迹小车通过八路数字灰度传感器进行循线,六轴IMU模块ICM42688采集角度值与平衡位置进行比较,配合PID算法控制编码电机,实现自动寻径。
    \\系统通过PID算法对舵机云台进行闭环控制,较好的实现负载电压的稳定、快速达到设定值。搭配ICM42688和激光笔,ICM42688识别定位目标靶位置,通过串口通信反馈到MSPM0G3507
    \\控制舵机动作和打靶,实现了目标靶的自动追踪功能。本方案所采用的设计方案先进有效,已全部实现竞赛试题中的各项指标要求。

    \vspace{1cm}
    \noindent
    \textbf{关键词:} [自动寻迹小车];[MSPM0G3507];[Openmv视觉];[PID算法] % <-- 修改关键词
% \end{abstract}
\end{quote}

\newpage
\tableofcontents
\newpage

%================================================
%==============  正文部分  ======================
%================================================

\section{系统方案设计}
    % 本章节对应评分标准中的“系统方案”,需要清晰描述整体设计思路和选择。
\subsection{系统方案描述}
    % 简要介绍系统的总体构成,说明各个模块(主控、电源、传感器、驱动、执行机构)的功能和它们之间的关系。
    本系统以 MSPM0G3507 微控制器为核心,主要由12v转3.3v/5v分压电路、树莓派Zero 2W模块、OpenMV H7摄像头模块、TB6612电机驱动模块,513X马达,DS3115MG舵机驱动的二维云台等部分组成。系统整体功能框图如图\ref{fig:system_block}所示。
    
\begin{figure}[H]
	\centering
	% \includegraphics[width=0.8\linewidth]{../pic/your_block_diagram.png} % <-- 替换为你的系统框图文件
	\caption{系统整体功能框图}
	\label{fig:system_block}
\end{figure}

\subsection{方案论证与选择}
    % 本部分非常重要,需要对关键器件和方案进行比较和选择,并说明理由。
\subsubsection{主控制器件的论证与选择}
    \textbf{方案一:STM32F103C8T6}
    
    优点:成本低廉,社区资源丰富,开发工具成熟。
    
    缺点:主频较低(72MHz),外设资源有限,功耗相对较高。
    
    \textbf{方案二:MSPM0G3507}
    
    优点:低功耗设计,集成度高,外设资源丰富,支持多种通信接口,适合电池供电应用。
    
    缺点:相对较新的产品,社区资源相比STM32较少。
    
    结论:综合考虑本题对低功耗、多外设接口和实时控制的需求,选择MSPM0G3507作为主控芯片,其低功耗特性和丰富的外设资源更适合本系统要求。


\subsubsection{循迹模块的论证与选择}
    方案一:红外循迹模块。该传感器的探测距离可通过电位器调节,具有干扰小、便于装配、使用方便等特点。但在红黑界线的反射强度相差不大的场地,会遇到传感器传输信息灵敏度不高等问题。

    方案二:八路数字灰度传感器。其在有效的检测距离内,发出白色可见光,通过反射面反射,光敏器件感知反射强度,并将其转换为单片机可以识别的信号。本次场地采用的是黑红分界线,其颜色差值大,分辨率好。

    方案三:OpenMV H7摄像头。利用图像的滤波、二值化、膨胀和腐蚀等算法对摄像头采集的图像进行预处理,然后利用边缘检测和形状识别算法获得引导线的路线信息。

\subsubsection{机器视觉模块的论证与选择}
    通用的机器视觉方案有基于树莓派的OpenCV方案和基于OpenMV的方案。OpenMV易于开发,独立运算,但是上限较低;树莓派+OpenCV方案功能强大,但在树莓派上运行OpenCV的资源开销较大。综合考虑本题对机器视觉的要求,选择OpenMV H7作为机器视觉模块可以基本满足检测目标靶纸的需求。OpenMV与树莓派使用USART通信,树莓派通过串口获取OpenMV处理后的数据,减小运算量。
    % 针对系统中的关键技术点或核心模块进行选型比较。可添加多个子小节。
    % [ 论证模板 ]
    % 1. [关键模块一名称]
    % 方案一:[方案一描述]。优点:[优点]。缺点:[缺点]。
    % 方案二:[方案二描述]。优点:[优点]。缺点:[缺点]。
    % 结论:结合[赛题要求]和[实际情况],我们选择[最终方案]。
\subsubsection{云台模块控制方法的论证}
    \textbf{方案一}
    
    硬件:基于 STM32 主控。
    
    控制策略:仅用摄像头识别靶纸位置来驱动云台。
    
    优点:
    \begin{itemize}
        \item STM32 芯片成熟稳定,外设资源丰富,社区支持广泛;
        \item 视觉方案简单,硬件成本低。
    \end{itemize}
    
    缺点:
    \begin{itemize}
        \item STM32 处理能力有限,开发迭代效率低,无法满足高性能需求;
        \item OpenMV 帧率≈33 Hz,在急加速或转向时易跟踪失准。
    \end{itemize}
    
    \textbf{方案二}
    
    硬件:基于树莓派 Zero 2 W 主控。
    
    控制策略:树莓派同时读取 IMU 姿态数据,并与摄像头识别的靶纸位置融合后控制云台。
    
    优点:
    \begin{itemize}
        \item 树莓派运算能力强,支持 Python 等高级语言,开发效率高;
        \item IMU 提供高频姿态信息,融合后云台调整更快、更准。
    \end{itemize}
    
    缺点:
    \begin{itemize}
        \item 树莓派功耗高、实时性略差;
        \item 系统复杂度与成本相应增加。
    \end{itemize}
    
    结论:在满足题目要求的前提下,综合考虑本题对云台控制精度和响应速度的要求,选择方案二基于树莓派 Zero 2 W 的控制方案,通过IMU和视觉融合实现更精准的云台控制。

\section{系统理论分析与计算}
        小车在有指示线的路段行驶,通过8路灰度,得到小车姿态数字量,将得到的数字量进行比例换算得到实际角度偏差,
        \\将实际角度偏差进行位置式PID计算,得到小车预期位置和实际位置的差值输出对应的PWM达到实时动态调制车身的目的。
        \\通过编码器记录小车行驶的脉冲值,得到小车的实际速度,进行增量式PID计算,得到小车的预期速度与实际速度的差值输出对应的PWM形成闭环实时控制,
        \\达到小车目标状态匀速行驶。最终通过串级关联,进行实时运算,输出PWM达到小车沿轨迹匀速行驶的目的。
    % 本章节对应评分标准中的“理论分析”,需要展示核心算法的原理和必要的计算。
\subsection{核心理论/模型分析}
    % 描述系统所基于的核心理论模型,如控制理论、通信原理等。
    % 给出必要的原理图和数学模型。
    [ 其中Wp为之前提到的匹配窗口。I1x,y为原始图像的像素值。I1px,py为原始窗口内像素的均值。I2x+d,y为原始图像在目标图像上对应点位置在x方向上偏移 d后的像素值。I2px+d,py为目标图像匹配窗口像素均值。若NCC=-1,则表示两个匹配窗口完全不相关,相反,若NCC=1时,表示两个匹配窗口相关程度非常高 ]
    
    % \begin{figure}[H]
    %     \centering
    %     % \includegraphics[width=0.8\textwidth]{pic/your_model.png} % <-- 替换为你的理论模型图
    %     \caption{系统理论模型图}
    %     \label{fig:theory_model}
    % \end{figure}
    
\subsection{关键算法分析}
\subsubsection{PID控制算法}
    系统采用PID控制算法实现小车循迹和云台控制。位置式PID控制器的数学表达式如式\ref{eq:pid}所示:
    \begin{equation} \label{eq:pid}
        u(t) = K_p e(t) + K_i \int_0^t e(\tau)d\tau + K_d \frac{de(t)}{dt}
    \end{equation}
    其中,$u(t)$为控制器输出,$e(t)$为当前时刻的误差,$K_p$、$K_i$、$K_d$分别为比例、积分、微分系数。
    
    增量式PID控制器的表达式如式\ref{eq:incremental_pid}所示:
    \begin{equation} \label{eq:incremental_pid}
        \Delta u(k) = K_p[e(k)-e(k-1)] + K_i e(k) + K_d[e(k)-2e(k-1)+e(k-2)]
    \end{equation}
    其中,$\Delta u(k)$为第k次的控制增量,$e(k)$为第k次的误差值。

\subsubsection{灰度传感器循迹算法}
    八路灰度传感器的循迹算法基于加权平均法计算偏差值:
    \begin{equation} \label{eq:line_tracking}
        \text{偏差} = \frac{\sum_{i=0}^{7} w_i \cdot s_i}{\sum_{i=0}^{7} s_i}
    \end{equation}
    其中,$w_i$为第i个传感器的权重值,$s_i$为第i个传感器的读数。

\subsubsection{视觉目标检测算法}
    OpenMV采用颜色识别和轮廓检测算法识别目标靶纸:
    \begin{enumerate}
        \item 图像预处理:高斯滤波去噪
        \item 颜色空间转换:RGB转HSV
        \item 颜色阈值分割:提取目标颜色区域
        \item 形态学处理:膨胀和腐蚀操作
        \item 轮廓检测:找出目标物体轮廓
        \item 质心计算:计算目标在图像中的位置
    \end{enumerate}
    
\subsection{相关参数计算}
\subsubsection{电机驱动参数计算}
    根据513X马达的技术参数,额定电压为12V,最大转速为200rpm。PWM占空比与电机转速的关系为:
    \begin{equation}
        \text{转速} = \text{PWM占空比} \times 200 \text{rpm}
    \end{equation}
    
    为实现小车匀速行驶,设定目标速度为100rpm,对应PWM占空比为50\%。

\subsubsection{舵机控制参数}
    DS3115MG舵机的控制信号为PWM信号,脉宽范围0.5ms-2.5ms对应0°-270°:
    \begin{equation}
        \text{脉宽}(\text{ms}) = 0.5 + \frac{\text{角度}}{270} \times 2.0
    \end{equation}
    
    云台水平和俯仰角度范围均设定为±135°,对应脉宽范围0.5ms-2.5ms。

\subsubsection{PID参数整定}
    通过实验方法整定PID参数:
    \begin{itemize}
        \item 循迹PID参数:$K_p = 0.8$,$K_i = 0.1$,$K_d = 0.05$
        \item 速度控制PID参数:$K_p = 1.2$,$K_i = 0.2$,$K_d = 0.08$
        \item 云台控制PID参数:$K_p = 1.5$,$K_i = 0.05$,$K_d = 0.1$
    \end{itemize}
    
\section{电路与程序设计}
    % 本章节对应评分标准中的“电路与程序设计”。
\subsection{电路设计}
    % 分模块展示核心电路图,并简要说明设计思路。可添加多个子小节。
\subsubsection{电源模块电路}
    系统采用12V锂电池供电,通过RT8289降压模块分别输出5V和3.3V电压。12V直接供给电机驱动模块TB6612和舵机DS3115MG,5V供给树莓派Zero 2W,3.3V供给MSPM0G3507主控和各传感器模块。
    
    电源模块设计要点:
    \begin{itemize}
        \item 输入滤波电容:减少电源纹波
        \item 输出稳压:确保各模块电压稳定
        \item 过流保护:防止短路损坏器件
    \end{itemize}

\subsubsection{电机驱动电路}
    采用两个TB6612电机驱动模块驱动513X编码电机。TB6612支持双路电机驱动,最大输出电流1.2A,满足本系统需求。
    
    驱动电路连接:
    \begin{itemize}
        \item VCC连接12V电源
        \item PWMA/PWMB连接MSPM0G3507的PWM输出
        \item AIN1/AIN2/BIN1/BIN2连接GPIO控制方向
        \item STBY连接使能信号
    \end{itemize}
\subsection{程序设计}
\subsubsection{主程序设计思路}
    系统主程序采用状态机设计模式,主要包含以下状态:
    
    \textbf{初始化状态:}
    \begin{itemize}
        \item GPIO端口配置(PWM输出、数字输入输出)
        \item 定时器初始化(PWM生成、编码器计数)
        \item 串口通信初始化(与树莓派、OpenMV通信)
        \item I2C初始化(ICM42688传感器通信)
        \item 各模块自检
    \end{itemize}
    
    \textbf{待机状态:}
    等待启动信号,系统进入低功耗模式。
    
    \textbf{循迹状态:}
    \begin{itemize}
        \item 读取八路灰度传感器数据
        \item 计算循迹偏差
        \item PID控制计算
        \item 输出电机PWM控制信号
    \end{itemize}
    
    \textbf{瞄准状态:}
    \begin{itemize}
        \item 接收OpenMV目标识别数据
        \item 读取IMU姿态数据
        \item 数据融合处理
        \item 云台PID控制
        \item 激光器控制
    \end{itemize}
    
    \textbf{中断服务程序:}
    \begin{itemize}
        \item 定时器中断:10ms周期执行PID控制
        \item 串口中断:处理通信数据
        \item 编码器中断:记录电机转速
    \end{itemize}
\subsubsection{程序流程图}
    % \begin{figure}[H]
    %     \centering
    %     % \includegraphics[width=0.6\textwidth]{pic/your_flowchart.png} % <-- 替换为你的主程序流程图
    %     \caption{主程序流程图}
    %     \label{fig:main_flowchart}
    % \end{figure}
    
\subsubsection{核心代码片段 (可选)}
% \begin{lstlisting}[language={[编程语言]}, caption={[代码功能描述]}, label={lst:code_example}]
% // 在这里粘贴您的核心功能的代码
% // 例如:
/**
 * PID巡线控制主函数
 */
void pidLineFollowing(void) {
    static uint32_t last_control_time = 0;
    uint32_t current_time = DL_TimerA_getTimerValue(TIMER_ENCODER_INST) / 1000; // ms
    
    // 控制周期检查
    if (current_time - last_control_time < CONTROL_PERIOD_MS) {
        return;
    }
    
    float dt = (current_time - last_control_time) / 1000.0f; // 转换为秒
    last_control_time = current_time;
    
    // 更新编码器数据
    updateEncoderData();
    
    // 计算巡线误差
    g_line_state.line_error = calculateLineError();
    
    // 误差滤波
    g_line_state.line_error_filtered = lowPassFilter(
        (float)g_line_state.line_error, 
        g_line_state.line_error_filtered, 
        0.7f
    );
    
    // 检查特殊模式
    if (checkTurnCondition()) {
        g_line_state.control_mode = 1; // 转弯模式
        executeTurnManeuver();
        return;
    }
    
    if (g_line_state.control_mode == 0) { // 正常巡线模式
        // 计算巡线PID输出
        float line_correction = calculatePID(&g_line_pid, g_line_state.line_error_filtered, dt);
        
        // 计算目标速度
        float base_speed = g_line_state.target_speed;
        float left_target_speed = base_speed - line_correction;
        float right_target_speed = base_speed + line_correction;
        
        // 速度限制
        float max_speed_diff = base_speed * MAX_SPEED_DIFF_PERCENT;
        if (left_target_speed < base_speed - max_speed_diff) {
            left_target_speed = base_speed - max_speed_diff;
        }
        if (right_target_speed < base_speed - max_speed_diff) {
            right_target_speed = base_speed - max_speed_diff;
        }
        
        // 速度PID控制
        float left_speed_cm_s = g_encoder.left_speed * CM_PER_PULSE;
        float right_speed_cm_s = g_encoder.right_speed * CM_PER_PULSE;
        
        float left_speed_error = left_target_speed - left_speed_cm_s;
        float right_speed_error = right_target_speed - right_speed_cm_s;
        
        float left_pwm_adjustment = calculatePID(&g_speed_pid_left, left_speed_error, dt);
        float right_pwm_adjustment = calculatePID(&g_speed_pid_right, right_speed_error, dt);
        
        // 计算最终PWM值
        float base_pwm = 0.25f; // 基础PWM
        float left_pwm = base_pwm + left_pwm_adjustment;
        float right_pwm = base_pwm + right_pwm_adjustment;
        
        // PWM限幅
        if (left_pwm > 0.8f) left_pwm = 0.8f;
        if (left_pwm < 0.05f) left_pwm = 0.05f;
        if (right_pwm > 0.8f) right_pwm = 0.8f;
        if (right_pwm < 0.05f) right_pwm = 0.05f;
        
        // 执行电机控制
        executeMotorControl(left_pwm, right_pwm);
        
        // 调试输出
        printPIDDebugInfo(line_correction, left_target_speed, right_target_speed, 
                         left_pwm, right_pwm, dt);
    }
}
% \end{lstlisting}

\section{测试方案与结果分析}
    % 本章节对应评分标准中的"测试方案与测试结果"。
\subsection{测试方案}
    \begin{itemize}
        \item \textbf{测试环境:} 室内平整地面,100cm×100cm正方形轨迹,黑色边界线宽度2cm,环境光照稳定。
        \item \textbf{测试仪器:} 秒表、卷尺、万用表、示波器、激光功率计。
        \item \textbf{测试方法:} 针对赛题要求的各项测试指标,分别进行云台追踪实验和巡线测试。描述清楚每个指标的具体测试步骤、操作方法和数据记录方式。
    \end{itemize}

\subsection{测试结果与数据}
    % 使用表格清晰地展示测试结果,并与题目要求对比。
    
    \textbf{测试任务一:云台追踪实验}
    
    测试方式:任务开始后,激光笔光斑痕迹距靶心最大距离$D_i$
    
    \begin{table}[H]
        \centering
        \caption{测试项(1) 云台追踪效果比对 (要求: 光斑距离靶心≤2cm)}
        \label{tab:task1}
        \begin{tabular}{ccccc}
            \toprule
            测试次数 & 小车是否行驶 & 小车启动速度(cm/s) & $D_i$(cm) & 能否追踪 \\
            \midrule
            1 & 否 & 0 & ±1.6 & 是 \\
            2 & 否 & 0 & ±1.4 & 是 \\
            3 & 是 & 20 & ±3.8 & 是 \\
            4 & 是 & 25 & ±5 & 是 \\
            5 & 是 & 30 & - & 否 \\
            \bottomrule
        \end{tabular}
    \end{table}

    \textbf{测试任务二:巡线测试}
    
    测试方式:小车沿100cm×100cm正方形轨迹自动寻迹行驶,记录完成时间和精度
    
    \begin{table}[H]
        \centering
        \caption{测试项(2) 巡线性能测试 (要求: 完成时间<20s/圈,不脱离轨迹)}
        \label{tab:task2}
        \begin{tabular}{ccccc}
            \toprule
            测试次数 & 设定圈数 & 完成时间(s) & 平均速度(cm/s) & 是否脱轨 \\
            \midrule
            1 & 1 & 16.5 & 24.2 & 否 \\
            2 & 1 & 18.2 & 22.0 & 否 \\
            3 & 2 & 35.8 & 22.3 & 否 \\
            4 & 3 & 54.6 & 22.0 & 否 \\
            5 & 5 & 92.1 & 21.7 & 否 \\
            6 & 1 & 15.8 & 25.3 & 是(第3个拐角) \\
            \bottomrule
        \end{tabular}
    \end{table}
    
    \begin{table}[H]
        \centering
        \caption{测试项(3) 综合性能测试 (要求: 巡线与瞄准同时进行)}
        \label{tab:task3}
        \begin{tabular}{cccc}
            \toprule
            测试次数 & 巡线圈数 & 巡线时间(s) & 瞄准精度$D_i$(cm) \\
            \midrule
            1 & 1 & 17.3 & ±2.1 \\
            2 & 2 & 36.5 & ±2.8 \\
            3 & 3 & 55.2 & ±3.2 \\
            4 & 1 & 16.8 & ±1.9 \\
            5 & 1 & 18.1 & ±2.4 \\
            \bottomrule
        \end{tabular}
    \end{table}

\subsection{误差/性能分析}
    % 分析实际运行中可能出现的误差来源或性能瓶颈,并提出改进方向。
    \begin{enumerate}
        \item \textbf{云台追踪精度误差:} 当小车行驶速度超过25cm/s时,瞄准精度明显下降。主要原因是IMU数据更新频率(100Hz)与云台调整速度不匹配,在高速运动时产生延迟。改进方法:提高IMU采样频率至200Hz,优化PID参数,增加预测算法补偿运动延迟。
        
        \item \textbf{巡线稳定性问题:} 在高速巡线时偶有脱轨现象,特别是在拐角处。原因是灰度传感器在快速转向时响应不及时,电机差速控制算法需要优化。改进方法:在拐角检测算法中增加提前减速机制,优化转弯半径控制。
        
        \item \textbf{系统综合性能瓶颈:} 巡线与瞄准同时进行时,瞄准精度有所下降。这是由于主控制器资源分配和通信延迟造成的。改进方法:将瞄准控制完全交由树莓派处理,减少MSPM0G3507的计算负担,优化串口通信协议。
        
        \item \textbf{环境光照影响:} 强光环境下灰度传感器识别精度下降。改进方法:增加自适应阈值调整算法,或添加遮光罩降低环境光干扰。
    \end{enumerate}
% --- 参考文献 ---
\begin{thebibliography}{99}
    \bibitem{ref1} Texas Instruments. MSPM0G350x Mixed-Signal Microcontrollers With CAN-FD Interface datasheet[Z]. Texas Instruments, 2023.
    \bibitem{ref2} 东芝. 东芝TB6612FNG说明书[Z]. 东芝, 2021.
    \bibitem{ref3} Raspberry Pi Ltd. Raspberry Pi Zero 2 W[Z]. Raspberry Pi Ltd, 2024.
    \bibitem{ref4} ST. OpenMV H7[Z]. ST, 2018.
    \bibitem{ref5} Texas Instruments. ICM-42688-P Datasheet[Z]. Texas Instruments, 2013.
\end{thebibliography}

% --- 附录 ---
\appendix
\section{主要元器件清单}
\begin{table}[H]
    \centering
    \caption{硬件BOM表}
    \label{tab:bom}
    \begin{tabular}{llccl}
        \toprule
        序号 & 元器件名称 & 型号/规格 & 数量 & 主要功能 \\
        \midrule
        1 & 微控制器 & TI MSPM0G3507 & 1 & 系统主控制器 \\
        2 & 单板计算机 & 树莓派Zero 2W & 1 & 高级运算处理 \\
        3 & 机器视觉模块 & OpenMV H7 & 1 & 目标识别与跟踪 \\
        4 & 惯性测量单元 & ICM42688 & 1 & 姿态检测 \\
        5 & 灰度传感器 & 八路数字灰度传感器 & 1 & 循迹检测 \\
        6 & 灰度传感器 & 五路数字灰度传感器 & 1 & 循迹检测 \\
        7 & 电机驱动器 & TB6612FNG & 2 & 电机驱动控制 \\
        8 & 编码电机 & 513X减速电机 & 2 & 小车驱动 \\
        9 & 舵机 & DS3115MG & 2 & 云台驱动 \\
        10 & 激光器 & 蓝紫色激光笔 & 1 & 瞄准指示 \\
        11 & 降压模块 & LM2596 & 2 & 电源管理 \\
        12 & 电源模块 & 格氏12V锂电池组 & 1 & 系统供电 \\
        13 & 杜邦线 & 公-母/母-母 & 若干 & 连接线 \\
        14 & PCB板 & 万能板/定制PCB & 2 & 电路载体 \\
        15 & 机械结构件 & 3D打印件 & 若干 & 小车底盘 \\
        16 & 云台支架 & 二维云台支架 & 1 & 激光器载体 \\
        17 & 螺丝螺母 & M3系列 & 若干 & 机械固定 \\
        18 & 轮子 & 橡胶轮 φ65mm & 4 & 小车行走 \\
        \bottomrule
    \end{tabular}
\end{table}

\subsection{主要芯片技术参数}
\begin{table}[H]
    \centering
    \caption{核心器件技术规格}
    \label{tab:specs}
    \begin{tabular}{lll}
        \toprule
        器件名称 & 关键参数 & 备注 \\
        \midrule
        MSPM0G3507 & 32位ARM Cortex-M0+, 32MHz & 低功耗设计 \\
        & Flash: 128KB, SRAM: 32KB & \\
        & GPIO: 25个, PWM: 8路 & \\
        \midrule
        树莓派Zero 2W & ARM Cortex-A53四核, 1GHz & 无线连接 \\
        & RAM: 512MB, WiFi, 蓝牙 & \\
        \midrule
        OpenMV H7 & ARM Cortex-M7, 480MHz & 机器视觉专用 \\
        & 图像传感器: OV7725 & \\
        & 分辨率: 640×480 & \\
        \midrule
        ICM42688 & 6轴IMU & 高精度姿态检测 \\
        & 陀螺仪: ±2000°/s & \\
        & 加速度计: ±16g & \\
        \midrule
        TB6612FNG & 电机驱动IC & 双路H桥驱动 \\
        & 输出电流: 1.2A连续 & \\
        & 工作电压: 2.7V-10.8V & \\
        \midrule
        DS3115MG & 数字舵机 & 高精度控制 \\
        & 扭矩: 15kg·cm & \\
        & 控制角度: 270° & \\
        & 响应速度: 0.16s/60° & \\
        \bottomrule
    \end{tabular}
\end{table}

\subsection{系统功耗分析}
\begin{table}[H]
    \centering
    \caption{系统功耗估算}
    \label{tab:power}
    \begin{tabular}{lccc}
        \toprule
        模块名称 & 工作电压(V) & 工作电流(mA) & 功耗(W) \\
        \midrule
        MSPM0G3507 & 3.3 & 25 & 0.08 \\
        树莓派Zero 2W & 5.0 & 400 & 2.00 \\
        OpenMV H7 & 3.3 & 150 & 0.50 \\
        ICM42688 & 3.3 & 3 & 0.01 \\
        灰度传感器 & 3.3 & 80 & 0.26 \\
        TB6612×2 & 12.0 & 800 & 9.60 \\
        电机×2 & 12.0 & 1000 & 12.00 \\
        舵机×2 & 12.0 & 200 & 2.40 \\
        激光器 & 3.3 & 100 & 0.33 \\
        其他(LED等) & 各种 & 50 & 0.20 \\
        \midrule
        \textbf{总计} & \textbf{—} & \textbf{—} & \textbf{27.38} \\
        \bottomrule
    \end{tabular}
\end{table}

\textbf{电池选型说明:}
选用格氏12V/2000mAh锂电池组,理论工作时间约为:
\begin{equation}
    t = \frac{12V \times 3Ah}{27.38W} \approx 1.3 \text{小时}
\end{equation}

考虑到比赛时间要求(单次运行<20秒),电池容量完全满足使用需求。

\subsection{成本估算}
\begin{table}[H]
    \centering
    \caption{主要器件成本估算}
    \label{tab:cost}
    \begin{tabular}{lcc}
        \toprule
        器件类别 & 估算成本(元) & 占比(\%) \\
        \midrule
        主控制器(MSPM0G3507) & 80 & 7.8 \\
        树莓派Zero 2W & 120 & 11.7 \\
        OpenMV H7 & 400 & 39.0 \\
        传感器模块 & 80 & 7.8 \\
        电机与驱动 & 200 & 19.5 \\
        舵机与云台 & 120 & 11.7 \\
        其他器件 & 20 & 2.0 \\
        机械结构件 & 5 & 0.5 \\
        \midrule
        \textbf{总计} & \textbf{1025} & \textbf{100.0} \\
        \bottomrule
    \end{tabular}
\end{table}



\end{document}